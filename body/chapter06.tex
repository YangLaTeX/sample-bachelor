% !TEX root = ../main.tex

\chapter{内容XXX}

\section{引言}

每章的引言起到承接上一章引启下一章的作用。

\ldots\ldots

\section{定理和定义等}

\begin{theorem}[\cite{ren2010}]
宇宙大爆炸是一种爆炸。
\end{theorem}
\begin{definition}[(霍金)]
宇宙大爆炸是一种爆炸。
\end{definition}
\begin{assumption}
宇宙大爆炸是一种爆炸。
\end{assumption}
\begin{lemma}
宇宙大爆炸是一种爆炸。
\end{lemma}
\begin{corollary}
宇宙大爆炸是一种爆炸。
\end{corollary}
\begin{exercise}
宇宙大爆炸是一种爆炸。
\end{exercise}
\begin{problem}[(Albert Einstein)]
宇宙大爆炸是一种爆炸。
\end{problem}
\begin{remark}
宇宙大爆炸是一种爆炸。
\end{remark}
\begin{axiom}[(爱因斯坦)]
宇宙大爆炸是一种爆炸。
\end{axiom}
\begin{conjecture}
宇宙大爆炸是一种爆炸。
\end{conjecture}

\section{XXXX分析}

\subsection{算法}

算法不在规范中要求,此处不给出示例,在hitszthesis.sty中有定义示例。

\subsection{脚注}

不在再规范\footnote{规范是指\PGR\ 和\UGR}中要求,模板默认使用清华大学的格式。

\subsection{源码}

也不在再规范中要求。如果有需要最好使用minted包,但在编译的时候需要添加“-shell-escape”选项且安装pygmentize软件,这些不在模板中默认载入,如果需要自行载入。

\section{本章小结}

总结本章的叙述内容。

\lipsum[3]
